\section{Related Work}

In this section, we first review previous works related to uncertain flow visualization. Then we give the background and related works about particle filtering.

\textbf{Uncertain Flow Visualization.} One group of the methods focused on visualizing the local uncertainty. This class of algorithms treated uncertainty within the flow field as a local phenomenon. Glyph-based approaches were discussed in~\cite{citeulike:4002316, conf/visualization/LodhaPSW96}, and texture-based methods were proposed in~\cite{botchen:2006:IVUF, 10.1109/VIS.2005.97}. In~\cite{zuk:2008:UBVF}, a method was introduced to visualize uncertainty in bidirectional vector fields. Edge Map was proposed by \cite{10.1109/TVCG.2011.265} to visualize flow fields with quantified spatial and temporal errors. \cite{conf/visualization/SandersonJK04} presented a reaction-diffusion model to visualize the uncertainty. There exist approaches that were designed to visualize global uncertainty which is transported within the flow, such as topology based approaches presented in~\cite{Otto10a, Otto11a}. However, none of the approaches above focused on solving the uncertain particle tracing problem, which is still one of the most popular flow visualization techniques.

\textbf{Particle Filtering.} In this work, we make use of a well-known feature tracking framework called particle filtering~\cite{doucet2001sequential}, which has been successfully used in computer vision and medical image analysis. Many problems in these fields involve tracking unknown quantities from some given observations, such as road tracking and fiber tracking. Generally, prior knowledge has been modeled for these phenomenon and Bayesian models can be formulated with prior distributions for the target of interest and likelihood functions for the observations. Then the tracking problem can be solved based on posterior distributions. The particle filter provides a convenient and attractive approach to evaluate the posterior distributions without any constraint about the data. It has been developed over decades and been successfully applied to solve many real world problems. \cite{bb69534, journals/pami/GemanJ96} applied this approach for human body gestures and road tracking. \cite{Brun02whitematter, bjornemoMICCAI02} introduced particle filtering for probabilistic fiber tracking, which was then improved by~\cite{journals/mia/PontabryROSKD13, Zhang20095}. In visualization community, Zhang et al. also applied this method for tracking the movement of dynamic voxels~\cite{Zhao2012}.
